\subsection{Design Guidelines} 
TODO: This subsection outlines and discusses XX design guidelines. The first four guidelines concern providing accurate and accessible information about actionable suggestions in the behavior change process. Guidelines 5 to 8 address the design part about fostering motivations (and engagement) in the behavior change process. \citet{Schultz2002} distinguishes three types of knowledge in pro-environmental actions. Procedural knowledge is about the where, when, and how of some task. Impact knowledge is an individual?s beliefs about the consequence of some task. Normative knowledge is beliefs about behaviors of others. An information-based intervention design can provide information that aim to increase all the three types of knowledge. Guidelines 1-3, 4, and 6 address the three types of knowledge correspondingly.

\vspace{.3cm}
\noindent\textbf{1. Develop and enhance consumers' energy conservation know-how through action suggestions that are implementable in everyday life.}

Action suggestions are recommendations and tips for energy conservation actions. Besides the literature support stated in Subsection 2.1, the results of user studies at the Italian and Swedish test sites of the CIVIS project also suggest that receiving implementable suggestions from a reliable source would be useful for the households. Hence, we recommend to provide action suggestions that can be easily incorporated into everyday practices. In particular, this means that (1) if possible, make action suggestions inexpensive micro-actions or divide a complex action into smaller steps; (2) the suggestions shall be tailored to the local everyday context, and (3) the suggestions shall be provided (to consumers) in an easily accessible manner. 

\vspace{.3cm}
\noindent\textbf{2. Explicitly express action suggestions with concrete and reliable content.}

The complexity of the information presented, the framing of the message, and the credibility of the source are among the key issues in delivering effective information  \citep{Schultz2002}. 
 \citep{Abrahamse2005} propose to explicitly mention the intervention strategy and specify its exact content and which behaviors are targeted; the benefit is twofold: the specifications (1) can provide clear information and suggestions to consumers, and (2) can be used by researchers as a decisive factor in evaluating an intervention's (in)effectiveness. The Italian participants in the user group studies raised the concern that they often found themselves plunged in a series of conflicting advice from various sources. Therefore it is important for them to have a reliable source of information and suggestions on energy conservation strategies or actions. Similarly, some energy managers in the Hammarby Sj�stad housing cooperatives had bad experiences of interactions with energy providers offering energy efficiency services and companies selling energy efficient technologies, and they perceived these companies to have vested interests. The perceived reliable sources are e.g. national and international energy authorities, consumer and environmental organisations, electricity suppliers  \citep{CEER2015} as well as neighbours and friends, in contrast to salespeople  \citep{Selvefors2015}.

\vspace{.3cm}
\noindent\textbf{3. Provide suggestions range from one-time actions to routine actions.}

One-time actions (or one-shot behaviors) refer to efficiency (increasing) behaviors, many of which entail the purchase of energy-efficient equipments  \citep{Abrahamse2005,Gardner2008} e.g. using a fridge with A+++ energy label, and installation of attic insulation. Routine actions refer to curtailment behaviors that involve repetitive efforts to use equipments less frequently or intensively  \citep{Abrahamse2005,Gardner2008}, e.g. thawing food in the refrigerator, and air-drying clothes. On the one hand, one-time actions often require purchasing, which offsets their advantage of simplicity, whereas most routine actions have no financial cost  \citep{Abrahamse2005,Gardner2008}. On the other hand, one-time actions are often more beneficial and cost-effective in the long-term  \citep{Froehlich2009}, and their energy-saving potential is generally considered to be greater than that of routine actions  \citep{Abrahamse2005,Gardner2008}. While many interventions targeted at households aim to change routine practices  \citep{Froehlich2009}, we recommend to provide suggestions that range from one-time actions to routine actions. There are actions that are in-between one-time and routine, such as occasionally vacuuming behind the fridge and regularly defrosting the freezer.

\vspace{.3cm}
\noindent\textbf{4. Indicate the effort entailed by a suggested action and its potential impact in an understandable way.}

The potential benefits (or outcomes) of an action, and the practicality and convenience (or inconvenience) of performing the action are important for people's decisions on adopting and sustaining the action  \citep{Schultz2002,Claudy2013}. As discussed earlier, we recommend to provide actions suggestions that are practical and inexpensive so that they can be implemented in busy everyday life, and to include actions that range from one-time to routine actions, as the former has long-term benefits and the latter can be performed straightaway for energy conservation without purchasing. Such information (i.e., both the procedural information and impact information) should be presented to consumers in an easily understandable way. 

General consumers often have difficulties understanding energy presented in kilowatt
hours or water in cubic centimetres  \citep{Froehlich2009}. These technical units of measurement can be used when needed, while people often prefer to have explanatory information e.g. showing energy use as number of laptops and CO$_{2}$ exhaust as number of trees  \citep{Petkov2011}. Many studies report that energy conservation outcomes expressed in terms of monetary savings result in underestimation of the impact of the efforts to reduce consumption  \citep{Froehlich2009,Pierce2010,Abrahamse2013}. We recommend to express the effort and impact of each action in an easy and understandable way, for example, in a scale of one to five. This also makes the suggested actions easily comparable. 
Impact of actions that have already been taken can also be shown by comparing energy use before and after the action was taken. 

\vspace{.3cm}
\noindent\textbf{5. Enhance and maintain intrinsic motivation; promote more active and volitional forms of extrinsic motivation.}

Intrinsic motivation is defined as the doing of an activity for its inherent satisfactions (rather than for its presumed instrumental value); contrarily, extrinsic motivation is the doing of an activity to attain some separable outcome or consequence  \citep{Ryan2000}. In the context of energy conservation, intrinsic motivators of actions are e.g. pro-environmental values, and common well-being; extrinsic motivators of actions are e.g. monetary incentives, tangible rewards, competitions, and social pressure. 

A large body of research favors intrinsic motivation over extrinsic motivation for the following two main reasons. First, intrinsic motivation is more likely to result in long-term behaviour change compared to extrinsic motivation  \citep{He2010}. That is, extrinsic motivators can motivate energy conservation, particularly for one-time behaviors; however, behaviors that must be repeated (i.e. routine behavior) will likely stop once the external motivator is removed; extrinsic motivators may even inadvertently increase self-centered behaviors over pro-environmental behaviors  \citep{Swim2014}. Second, intrinsic motivation will lead to positive spillover of pro-environmental behaviors while extrinsic motivation will lead to negative spillover; positive or negative spillover refers to the effect that one pro-environmental behavior increases or decreases the likelihood of additional pro-environmental behaviors  \citep{thogersen2009simple,Truelove2014,Knowles2014}.

In situations where intrinsic motivations are low or absent, \citet{Ryan2000} propose to promote more active and volitional (versus passive and controlling) forms of extrinsic motivation  \citep{Ryan2000}.  \citep{Ryan2000} suggest that extrinsic motivation can vary greatly in the degree to which it is autonomous, i.e., one can perform extrinsically motivated actions with resentment, resistance and disinterest or, alternatively, with an attitude of willingness that
reflects an inner acceptance of the value or utility of a task.

For a high level of intrinsic motivation to be maintained or enhanced, or for extrinsic motivation to be more active and volitional, people must experience satisfaction of both the needs for (1) feelings of competence, and (2) senses of autonomy; this means that people must not only experience perceived competence (or self-efficacy), they must also experience their behavior to be self-determined (i.e. free choice rather than being controlled)  \citep{Ryan2000}. In such cases, an individual has a strong internal locus of control. Feelings of competence can be enhanced e.g. through positive performance feedback and encouragement of small steps or micro-actions, whereas senses of autonomy can be enhanced e.g. through allowing and facilitating people's own choices of taking up actions. In addition,  \citep{Ryan2000} suggest that the satisfaction of the needs for (3) senses of relatedness facilitates active and volitional extrinsic motivation (i.e. belongingness and connectedness to the person, group or culture disseminating a goal); intrinsic motivation possesses this condition by definition. Supports for relatedness and competence foster internalization, and supports for autonomy additionally foster integration of  values and behavioral regulations  \citep{Ryan2000}. We recommend to incorporate the facilitation of these three elements into the intervention design. 

\vspace{.3cm}
\noindent\textbf{6. Use social norms and public commitment to address low motivation.}
 
Normative knowledge (i.e. perceived social norms) is an understanding of the behavior of others  \citep{Schultz2002}. Descriptive social norms are beliefs about what other people are doing, often referred to as \textit{norms of is}, whereas injunctive social norms are beliefs about what other people think they should be doing, often referred to as \textit{norms of ought}  \citep{Schultz2002}. 
Research indicates that normative beliefs can predict a variety of behaviors, and normative interventions are effective in promoting pro-environmental behavior change by giving cues as to what is appropriate and desirable  \citep{Allcott2011,Schultz2002, Petkov2011,Delmas2013}. They are useful to address low motivation  \citep{schultz2015strategies}. 

Nonetheless, there are quite a few instances where normative beliefs would not be predictive, e.g. when one perceives that a behavior is desired but does not perceive that others are doing it and/or thinks the impact or benefits of one's own actions is very low (i.e. a strong external locus of control), or when one's behavior is not directly observable by other community members  \citep{Schultz2002,ockwell2009reorienting}. Many of these situations can be characterized as commons dilemmas (a.k.a. the tragedy of the commons  \citep{Hardin1968}), that is, whether to reduce one's individual rates of consumption, sacrificing their own desires, freedom to consume, and perhaps personal well-being for the future of the group, or to continue using the resources at the same rate for their own gain and with no regard for others, risking the common pool of resources  \citep{Edney1978,Edney1980}. Free riders are concrete examples of the commons problem. In energy consumption, free riders often appear when the energy cost is included in the rent (or utility package)  \citep{munley1990electricity} or when a residence has shared metering  \citep{dewees2011impact}.

Besides using private ownership and policy interventions to regulate this problem, communication can lead individuals to act in the interest of the group --- individuals are considerably more likely to reduce their use of the common when they believe that others who share access to the common will also limit their use  \citep{Edney1978,Schultz2002}. Public commitment (and disseminating this information)  \citep{mckenzie2000fostering,Abrahamse2005} is a promise or agreement made publicly by a person (or an organization, etc.) to perform a certain action or behavior. When one?s own behavior and that of others are publicly observable, the behavior is more likely to be affected by changes in normative beliefs, which in turn may contribute to tackling the commons problem  \citep{yim2011tale,Schultz2002}. Peer-pressure can induce cooperation among self-interested individuals such as free riders  \citep{mani2013inducing}. The user study in Helsinki suggests that people are willing to share publicly (or with a selected group of people) their energy conservation actions, and do not consider this a privacy issue. They are also interested in the conservation actions of others such as household members, neighbors, friends and similar consumers (and households).  Similarly, the housing cooperatives in the Hammarby Sj�stad pilot site thought it would be valuable to see building level energy reduction actions taken by other cooperatives and, since it is cooperative level information, they were also willing to share energy data for comparison purposes.


\vspace{.3cm}
\noindent\textbf{7. Facilitate consumers to reflect on (pro-environmental) lifestyle choices in the process of behavior change.}

\citet{Brynjarsdottir2012} critically reviewed ICT technologies designed for environmental behavior interventions (and persuasions). The authors point out that existing design, having a narrowed vision of sustainability, overly focuses on modernistic system change and individual consumption and entrusts designers with the responsibility to decide what is or is not appropriate behavior. They suggest to lessen the prescription of pro-environmental or sustainable actions chosen by designers, who may not connect with users' actual everyday life experiences, and instead to make design that help elicit issues of sustainability and encourage users for open-ended reflection on what it actually means to be sustainable in a way and with lifestyle choices that make sense in the context of their everyday life. With this goal, our design (1) lets users to choose and schedule the actions according to their needs and interests, and (2) facilitates commenting and discussions among users. 

\vspace{.3cm}
\noindent\textbf{8. Engage all household members.}

It is important to engage all household members in energy conservation in everyday life. The artifacts, technologies and resource systems to date are typically designed for ``household resource managers'', often men, although they are far from the only energy users in households \citep{Strengers2014}. Women dominate the everyday practices of the household (particularly cleaning activities), and are often more sensitive to understandings of presentability, body odour, hygiene and cosiness \citep{Strengers2014}. Women usually show more concern about environmental destruction, and are more emotionally engaged and willing to change \citep{Kollmuss2002}. Families with children generally consume more energy than those without and this consumption tends to increase as children grow older  \citep{Fell2014}. Children and teenagers are commonly recognized as lacking interest in energy bills, and they participate in or are the cause of many energy consuming practices \citep{Berthou2013,Strengers2014}. Studies show that children enjoy the involvement and responsibility in helping save energy, and parents? commitment also increases when they think about energy conservation in the context of their children's education  \citep{Burchell2014,Fell2014}. Discussing and establishing common family responsibilities around energy consumption is reported to be effective  \citep{huizenga2015shedding}. 

\vspace{.3cm}
\noindent\textbf{9. Support demand side management through simple time-of-use (ToU) signals.}

The overall efficacy of time-of-use tariffs for demand side management is still debated. However, dynamic
signals in energy grids with high presence of renewable energy sources (RES) that are characterized by
high unpredictability (\textit{e.g.} wind, solar, hydro with no basins) are worth considering \citep{Torriti2012,Palensky2011}.

To a great extent, supporting changes in energy demand -- from reduction to shifts -- implies supporting people in changing their daily routines, it does not only involve the
`simple' alternative use of an appliance. Energy is `used for something' and this `something` is always deeply rooted in people's habits and contexts \citep{Shove2014}. 
Recent literature claimed that interventions aimed at supporting demand side management by focusing on the \textit{appliance} level\footnote{This refers to automated appliances or semi-automated/remotely
controlled appliances.} of intervention, have very limited and only short-term implications.
More effective approaches work at the level of \textit{practice}: supporting people in
doing things differently \citep{Gruenewald2016}. Simplicity and clarity of means to support demand side management at the practice level are key factors to help people adapting and adpoting
alternative routines. In the particular case of ToU based approaches, \citet{Gruenewald2015} argued for a design of ToU signals that are the easiest to grasp by people. 


\vspace{.3cm}
\noindent\textbf{10. Define energy interventions through co-design approaches.}

As widely reported in literature\footnote{For an extensive overview (yet not exhaustive) of how societal aspects intersect behavioural and attitudes change in energy, see \citep{Owens2008}.},
changes of energy attitudes and behaviours are deeply rooted into the broader social environment in which they shall take place. For instance, failures to properly understand
the social constraints that are situated in the local contexts, greatly undermine the possibility for any policy, technology or form of intervention to be effective \citep{Devine-Wright2005}.
Pre-existing social norms, concrete and tangible local energy needs (which may well vary at municipality, regional or national levels) as well as perceived energy or
environmental challenges, they all impact households and people engagement in improving (sustainably) energy behaviours.

For an energy intervention to have chances of becoming locally relevant and sustainable in a given place, it is crucial to
engage with all the related stakeholders (\textit{e.g.} Public Administrations, DSOs, retailers, households, associations)
and to involve them in the definition and appopriation of such energy interventions -- in their objectives and, moreover, the tools and technologies to support them.
Codesign approaches are becoming more and more used to uncover the complexities of energy systems in their local contexts (\textit{e.g.} infrastructures, technologies, target user groups, energy needs)
and to design energy interventions that try to align, as much as possible, diverse sets of interests and expectations \citep{Dick2012,Tang2008}.
Energy interventions 
can become widespread and public matters of concerns \citep{DiSalvo2014} with increased sense of belonging and active
participation \citep{Throndsen2015}, if they are designed and deployed through the involvement of relevant stakeholders.  

Therefore processes for stakeholders engagement, practical means and channels for the elicitation and integration of their inputs
shall be put in place for the various phases of energy interventions: preliminary study and design,
development and testing, deployment and maintenance.   