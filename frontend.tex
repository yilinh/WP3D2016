\subsection{CIVIS Front-end as a Hybrid Application} 

The CIVIS front-end (YouPower) is developed as a hybrid (cross-platform) mobile application using \textit{Ionic}\footnote{\url{http://ionicframework.com/}}, an HTML5 front-end development framework built with SASS\footnote{\url{http://sass-lang.com/}} and optimized for AngularJS\footnote{\url{https://angularjs.org/}} (a.k.a Angular). 
% 
The Ionic framework comes with native-styled mobile UI elements and layouts, and handles the look and feel and the UI interactions the app needs in order to be compelling\footnote{\url{http://ionicframework.com/docs/guide/}}. 

Angular as a JS framework provides directives (extensions of HTML attributes) and two-way data binding (binds input or output data of the view to a model) that simplify the app development with Model-View-Controller (MVC) architecture. 
In two-way data binding, the value of a data model is passed on from the view (or loaded from the back-end) to the controller at run-time, and the function in the controller returns the result (of the value manipulation) to the view. 
% 
Other noteworthy JS and Angular libraries (i.e. besides Ionic) we use for the front-end development are as follows:

\begin{itemize}

\item Highcharts\footnote{\url{http://www.highcharts.com/}}, a charting library in JS. It provides an easy way to add interactive charts to the application. 

\item Highcharts-ng\footnote{\url{https://github.com/pablojim/highcharts-ng}}, a simple Angular directive for Highcharts. 

\item Angular-translate\footnote{\url{https://angular-translate.github.io/}}, an Angular module for internationalization and localization of the application. (YouPower is currently available in English, Swedish and Italian.)

\item Angular-resource\footnote{\url{https://docs.angularjs.org/api/ngResource}}, an Angular module for interacting with RESTful server-side data sources. 

\item Underscore.js\footnote{\url{http://underscorejs.org/}}, a JS library that provides over 100 utility functions. 

\item Bootstrap-sass\footnote{\url{https://github.com/twbs/bootstrap-sass}}, a Sass-powered version of Bootstrap 3. 

\item Moment.js\footnote{\url{http://momentjs.com/}}, a JS library to parse, validate, manipulate and display dates. 

\item Ion-datetime-picker\footnote{\url{https://www.npmjs.com/package/ionic-datetime-picker}}, a date and time picker for the ionic framework. 

\end{itemize}
\subsubsection{Feature improvement - technologies in frontend}
YouPower has been updated to give trentino users the capability to visualize their consumption and production data as well as comparing personal consumption and production with the benchmark. We have used the following list of modules and frameworks to develop the new feature. Description and use of each tool is given in section 3.1 above.
\begin{itemize}
\item Highcharts
\item highcharts-ng
\item Moment
\item ion-datetime-picker
\item angular-translate
\item Ionic
\end{itemize}
\subsubsection{Technical issues}
 The following table summarizes the technical issues we have faced in developing the new features including those which are already solved and integrated to YouPower and issues which will be fixed in upcoming updates.  
\begin{table}[h!]
\caption{Technical issues}\label{tab:app_nav}
\begin{center} \footnotesize 
\begin{tabular}{ l p{6cm}}
\hline
\textbf{Issue}  &
\textbf{Status}  \\ \hline

DatePicker incompatibility with firefox
  & 
Not fixed -- planned for next update\\ 
CORS issue when accessing CN endpoints & Fixed: CN server allow civisproject domain \\ 
Mixed content issue(secured and unsecured content) & Status: Fixed: We developed 2 endpoints in backend for accessing TOU signals(originally non-secured) from creatNet serverand provide secured content from YouPower backend.\\
 \hline
\end{tabular}
\end{center} 
\end{table}