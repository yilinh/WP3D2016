\subsection{CIVIS Front-end as a Hybrid Application} 

The CIVIS front-end (YouPower) is developed as a hybrid (cross-platform) mobile application using \textit{Ionic}\footnote{\url{http://ionicframework.com/}}, an HTML5 front-end development framework built with SASS\footnote{\url{http://sass-lang.com/}} and optimized for AngularJS\footnote{\url{https://angularjs.org/}} (a.k.a Angular). 
% 
The Ionic framework comes with native-styled mobile UI elements and layouts, and handles the look and feel and the UI interactions the app needs in order to be compelling\footnote{\url{http://ionicframework.com/docs/guide/}}. 

Angular as a JS framework provides directives (extensions of HTML attributes) and two-way data binding (binds input or output data of the view to a model) that simplify the app development with Model-View-Controller (MVC) architecture. 
In two-way data binding, the value of a data model is passed on from the view (or loaded from the back-end) to the controller at run-time, and the function in the controller returns the result (of the value manipulation) to the view. 
% 
Other noteworthy JS and Angular libraries (i.e. besides Ionic) we use for the front-end development are as follows:

\begin{itemize}

\item Highcharts\footnote{\url{http://www.highcharts.com/}}, a charting library in JS. It provides an easy way to add interactive charts to the application. 

\item Highcharts-ng\footnote{\url{https://github.com/pablojim/highcharts-ng}}, a simple Angular directive for Highcharts. 

\item Angular-translate\footnote{\url{https://angular-translate.github.io/}}, an Angular module for internationalization and localization of the application. (YouPower is currently available in English, Swedish and Italian.)

\item Angular-resource\footnote{\url{https://docs.angularjs.org/api/ngResource}}, an Angular module for interacting with RESTful server-side data sources. 

\item Underscore.js\footnote{\url{http://underscorejs.org/}}, a JS library that provides over 100 utility functions. 

\item Bootstrap-sass\footnote{\url{https://github.com/twbs/bootstrap-sass}}, a Sass-powered version of Bootstrap 3. 

\item Moment.js\footnote{\url{http://momentjs.com/}}, a JS library to parse, validate, manipulate and display dates. 

\item Ion-datetime-picker\footnote{\url{https://www.npmjs.com/package/ionic-datetime-picker}}, a date and time picker for the ionic framework. 

\end{itemize}