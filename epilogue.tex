\section{Epilogue}


This section is added to the original submission of D3.2 to address the review remarks from the Technical Review Report.  

\vspace{3mm}

\noindent\textbf{1. The D3.2 presents the You Power system as it was still under the first development phase \textit{at the moment of writing this report, YouPower is still in development. It is planned to be deployed at the Stockholm and Trentino test sites by Oct. 2015}). According to the presentation during 2nd review, the front end is mostly implemented and deployed within the test sites. The presented progress in not up-to-date;}
\vspace{3mm}

The content reported in D3.2 was prepared during Jul-Aug 2015. By the time of the review meeting in Brussels, the development of the CIVIS platform advanced. So we presented the latest state of the development as requested by the project officer. By November 19, 2015 (the date of the 2nd year review meeting), the first version of the Stockholm test site app was just deployed. The first version of the Trentino test site app was deployed in March 2016. The details will be reported in D3.3.
\vspace{3mm}

\noindent\textbf{2. The mock-up of the \textit{housing cooperatives} part of the CIVIS app was evaluated by managers but only one was in fact an energy manager [D5.2, page 37]. There is also lack of details in description of the housing cooperatives mock-up;}

\vspace{3mm}

The mock-up of the housing cooperatives part was discussed in a focus group with three representatives from different housing cooperatives and they were all working with energy issues in their cooperative. One of the cooperatives did not have any official energy manager role, although in practice the person who was garden responsible also managed the energy work. Another representative was no longer a member of the cooperative board but continued to voluntarily do the energy management work on behalf of the cooperative. Consequently, all three representatives in the mock-up focus group were working with similar issues but their official roles varied. The same will be the case for other potential users of the app; the cooperatives inevitably have to deal with energy issues but the person managing the energy work may not be called energy manager.

The main purposes of the housing cooperative part is to provide means for learning about energy reduction actions between housing cooperatives and within a housing cooperative over time. Details to add to the current description are that the learning is supported by mapping energy data, which is automatically updated in the app, with energy reduction actions that are added by energy managers (or others with similar responsibilities). When adding actions that a cooperative has taken, the user can categorise the action based on which type of action it was, write a description of the action and fill in information about the cost. To support communication between cooperatives there is also an option for each cooperative to add an email address to their energy contact person. Further details about the final housing cooperative app design will be provided in D3.3. 

\vspace{3mm}

\noindent\textbf{3. It is not clear which modules of features are or will be implemented and which are only the ideas (a predictive engine, crowdfunding campaign, visualization of the donation programme status);}

\vspace{3mm}

The predictive engine is already implemented and deployed to the Italian pilot area.
Visualization of the donation programme status is currently under development and will be deployed into to Italian pilot area. It will show individual and collective cumulative progresses of consumptions in relation to the signals provided by predictive engine (ToU signals). More details will be reported in D3.3. 

\vspace{3mm}

\noindent\textbf{4. It is advised that real-time consumption is also shown in the users' computers. In this way a
consumer may, for example, realize that he has an unusually high electric consumption because he forgot some load on.}

\vspace{3mm}

Real time consumption and production (for those who have PV at household level) has been already implemented. Visualization of real time consumption data will also be improved and refined based on end-users feedback. More details will be reported in D3.3. 

\vspace{3mm}

\noindent\textbf{5. The action suggestions given in paragraph 2.3.1 are too simple and obvious. The existence of the CIVIC platform should be fully exploited by giving the opportunity to a user to compare his real-time consumption with other relevant data such as aggregated DSO load, own production, etc.}

\vspace{3mm}

An important goal of providing the action suggestions mentioned in Subsection 2.3.1 is to facilitate the behavior change process in order to address the attitude-behavior gap. It is often reported in literature that despite growing environmental awareness and articulated preference for ``green'' lifestyles, people'schuls pro-environmental values and attitudes often fail to materialize in actual pro-environmental actions and behavior changes, from energy conservation, recycling to the purchase of green products \citep{Schultz2002,Abrahamse2005,Claudy2013}. This imparity is commonly referred to as the attitude-behavior gap or the value-action gap \citep{Blake1999,Kollmuss2002,Claudy2013}. This means that even when people know that certain actions (some of which may be simple and obvious) could help save energy, they are not actually doing them in daily practices. Our design aims to bridge this attitude-behavior gap. More details will be discussed in D3.3.

The CIVIS platform part related to real-time consumption and load-balancing are discussed in Subsection 2.3.3. 

\vspace{3mm}

\noindent\textbf{6. It is mentioned on page 27: \textit{One important aspect in terms of community is the ability to compare a single household?s consumption level with a benchmark value, in line with the principle of normative comparison (Huang and Miorandi, 2014; Cialdini and Schultz, 2004), which is part of the social norm approach frame outlined in the previous sections.} The question is the time-frame of the comparison and the availability of the respective benchmark. For example is it feasible to compare hourly values when the benchmark is available for monthly or daily periods? If the user is meant to take actions due to the comparison, the time frames should be as small as possible (eg, 1 hour or less). How will you create reliable benchmarks for such small periods?}

\vspace{3mm}

Due to the delays hoarded for the deployment of the first release to the Italian pilot area, at this point it is no longer feasible to implement a benchmarking system with the details and reliability we initially anticipated and to deliver it to the pilot area in time to collect enough information. Furthermore, the recruited sample of participants is too small to create `per-categories' and ad-hoc benchmarks. Therefore, implementation of such a normative benchmarking has been largely abandoned.

Works for a simpler benchmarking system (i.e. broader time frames and total user base) are still under consideration: comparison between a single user energy consumption and the average of the consortium they belong to. This will be reported in D3.3. 

