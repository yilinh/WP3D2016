% !TeX spellcheck = en_GB
\subsubsection{Housing Cooperatives (Hammarby Sj�stad, Stockholm Test Site)}
\label{sect:brf}

\paragraph{Design Concept}
The ``housing cooperatives'' part of the app (see Figure~\ref{fig:brf} for a mock-up) is considered for households in Hammarby Sj�stad in Stockholm. In Sweden, everybody who buys an apartment also joins a housing cooperative (bostadsr�ttsf�rening in Swedish). The cooperative owns the building and annually elects a board that is in charge of the finances and maintenance of the building, including energy related decisions and work. Some housing cooperatives in Hammarby Sj�stad have decided to make one of the board members in charge of energy issues and this role has been named energy manager (see the Hammarby Sj�stad use case in D1.2 for more information). Primary users of the housing cooperative part of the app are energy managers and board members of housing cooperatives, and secondary users are other housing cooperative members. There are three main categories of features: energy information about the user's own housing cooperative, energy information about other housing cooperatives, and support for communication between energy managers.

Housing cooperative energy information includes comparative energy performance (kWh /m$^{2}$)\footnote{We chose to use kWh in this case since cooperative managers have fairly good knowledge of energy units and the visualization is comparative.} and a bar chart with the cooperative's monthly or yearly energy use, divided into heating (including hot water) and facilities electricity. Linked to the chart, energy actions that have been taken are listed. The user can see when different actions, such as energy information, optimisation or investments, were previously taken and see more details about the actions. By comparing the energy use with previous periods, the user can also see the impact of the actions.

In the same way that the users can view information about their own cooperative, they can also see the energy performance and energy actions taken by other cooperatives. This allows energy managers, and others who are interested, to e.g. explore the effect of a neighbouring cooperative's actions on their energy use or read about how they carried out an investment and which contractor was used. 

To further support collaboration and knowledge exchange between housing cooperatives the app has a discussion group dedicated for energy managers in Hammarby S\"{o}stad. Within the group they have the possibility of creating discussion topics of their interest. In this way, the discussion of the occasional meetings with the local energy network can be extended to continue online.

\begin{figure}
\centering
\includegraphics[width=0.65\linewidth]{"img/housing cooperatives 1"}
\includegraphics[width=0.65\linewidth]{"img/housing cooperatives 2"}
\caption{Mock-up of the ``housing cooperatives'' part}
\label{fig:brf}
\end{figure}


\paragraph{Design Evaluation}

The mockup of the ``housing cooperatives'' part of the CIVIS app was evaluated with three energy managers in Hammarby Sj\"{o}stad and the feedback was incorporated in the design and development of the application. The energy managers would primarily want to use the app to find housing cooperatives with similar challenges and see what actions they had taken. They also thought the app would be helpful for deciding which companies can be trusted, based on what other housing cooperatives had done and what the effects were on the energy use. The energy managers doubted that other members in their cooperatives would be very interested in following the cooperative's energy use, but they though the app might be useful for engaging members in specific questions. 

