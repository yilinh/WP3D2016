
\subsection{Analytical Framework}

The CIVIS platform design has a strong theoretical basis and is tightly coupled to the Stockholm and Trentino use cases (see WP1 D1.2 report). 
The key analytical frames we used are summarized as follows. 

\begin{description}
\item[1. Making Energy Visible] Energy is consumed in many unconsidered daily practices and routines and is materially invisible \citep{Burgess2008,Hargreaves2010,Fehrenbacher2011,Burchell2014}. Methods and approaches to render energy visible (or energy conscious) can be complemented by the experiential and contextualized forms of learning \citep{Burchell2014}. 

\item[2. Behavior is Open to Change through Learning] The lack of energy know-how (i.e., knowing what to do and how to do it) is a significant constraint on behavior change \citep{Burchell2014}. For the experiential and contextualized forms of learning, \citet{Darby2006} proposes the construction of energy-related knowledge over time incorporating awareness (including explicit and tacit knowledge)\footnote{Awareness is ``a state of being alert and knowledgeable'' \citep{Darby2006}.}, (individual and community) action, and feedback (on consumption, processes and outcomes). 

\item[3. Social Norm Approach] The Social Norm Approach (SNA) has its roots in psychological and social psychological theories of conformity, which suggests that behavior can be shaped by social norms in a variety of forms \citep{Ayres2013,Abrahamse2013,Burchell2014,Dietz2015}. 

\item[4. Working Together in Everyday Life] Many studies in energy communities show that the sense of being a part of something is important to people \citep{Darby2006,Burchell2014}. The participation of people and households in energy-related actions is often constrained by the busyness and competing priorities of everyday life \citep{Burchell2014,Dillahunt2014,Strengers2014}. 
\end{description}

 %For example, a householder will not attempt to find out how to improve energy efficiency or conserve energy at home without a combination of the explicit knowledge that more affordable and/or less polluting energy services are possible and the tacit knowledge of how to being making the necessary changes \citet{Darby2006} .
 



