
\addcontentsline{toc}{section}{Executive Summary}
\section*{Executive Summary}

This report summarizes the 3rd year's activities of CIVIS WP3 and the results and contributions of the WP. The research attention of CIVIS is oriented towards the potentials and challenges of users' collective action, social values and sense of communities in smart grids. To this end, the main activities in the 3rd year of WP3 continued the iterative, theory-driven, and user-centered development process. WP3 completed the design and development of the CIVIS \textit{YouPower} platform, expanding upon the 2nd year's results. The platform is deployed and tested in the Stockholm and Trento test sites. The assessment evaluation of the platform for the purpose of social engagement is reported in D5.3 and the impact of energy behavior is reported in D7.3.

The design and development of YouPower can be divided into three self-contained, interrelated and composable parts: (1) Action Suggestions (contextualized and deployed to both test sites); (2) Housing Cooperatives (contextualized and deployed to the Swedish test site); and (3) Energy Awareness at Individual and Collective Levels (contextualized and deployed to the Italian test site). The intension of the application is to increase energy awareness, to inform users' energy consumption know-how, to promote environmental social norms, and to facilitate users for energy conservation and load-shifting behavior changes that are implementable in everylife. We derived ten design guidelines to address a number of design issues based on (pro-environmental interventions, social and environmental psychology) literature and our design experience. These guidelines are discussed in depth in this report and they can be useful for a wide group of environmental intervention designers: 
\begin{enumerate}
\item Develop and enhance consumers' energy conservation know-how through action suggestions
that are implementable in everyday life.
\item Explicitly express action suggestions with concrete and reliable content.
\item Provide suggestions ranging from one-time actions to routine actions.
\item Support demand side management (load-shifting) through simple time-of-use (ToU) signals.
\item Indicate the effort entailed by a suggested action and its potential impact in an understandable way.
\item Enhance and maintain intrinsic motivation; promote more active and volitional forms
of extrinsic motivation.
\item Use social norms and public commitment to address low motivation.
\item Facilitate consumers to reflect on (environmental) lifestyle choices in the process
of behavior change.
\item Engage all household members in energy conservation and in the process of behavior change.
\item Define energy interventions through co-design approaches.
\end{enumerate}

The key features designed for the YouPower platform include: 

\begin{enumerate}
\item Action suggestions, where users are provided with suggestions of actions that are implementable in everyday household practices. 
% 
It presents users with motivators for behavior change (and its process), and means for self-monitoring, self-reflection and social sharing of the pro-environmental behavior change process at individual, household and community levels. 
Each action is accompanied with the information about the potenial effort and impact, user comments and a list of participants who are taking this action. A user can decide whether to take an action and to schedule it. A user also see actions of household members. 

\item Housing cooperatives, where energy management and cooperative energy reduction actions are linked to each cooperative's energy performance. 
% 
The building level energy information and the energy related actions of each housing cooperative are presented to users. This allows comparison between housing cooperatives based on the energy performance and actions, and supports knowledge exchange and social sharing and learning among cooperatives in energy reduction actions.

\item Energy awareness at individual and collective levels, where users are presented with real time and historical energy consumption and production information. 
% 
It provides users with time-of-use signals based on real time and forecasted information about local production from renewable sources and the corresponding energy demand in order to enable load-shifting actions in everyday practice. It also presents users with reference information about energy performance in relation to the neighborhoods. 

\end{enumerate}

The CIVIS YouPower platform is currently deployed and tested at the Stockholm (Sweden) and Trento (Italy)
test sites corresponding to the local context. To reach a wide group of research communities and general public, the YouPower software is open source with an online repository and API documentation. The package management and application setup of YouPower are also discussed in this report to support potential usership and future development of the application. 

\newpage