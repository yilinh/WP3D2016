\section{Summary of WP3 Tasks Contribution}

\subsection{T3.1 Community Management}

Task 3.1 works in close cooperation with WP5 to explore community management. This is reflected in the diverse CIVIS platform functionalities in supporting housing cooperative activities in the Swedish test site (see Sec. \ref{sec:brf}), and the two local Consortia in the Italian test site respectively (see Sec. \ref{sec:energydata}). Those functionalities are designed and tailored to the local context as following:   

\begin{itemize}
\item 
\end{itemize}

\subsection{T3.2 Energy Consumer Profiling}

\subsection{T3.3 Interface with System Level}

The task 3.3 of CIVIS WP3 took care of exploring the different models for interconnecting the system level developed in WP4, the Energy ICT Platform, with the user level subsystem developed in WP3.

This work, performed between T02 and T08 of CIVIS project, was strictly in connection with the one performed in D4.3, where the interconnection of the two subsystems has been implemented and has been documented in D4.1, where the different models of interconnection have been described.

Sec. 8.3 of CIVIS D4.1 contains the descriptions of three different models that could be used in order to implement the connection between User Level and System Level:

\begin{itemize}
\item Solution 1: single connection and native HiReply MS SQL DB (Sec. 8.3.1 of D4.1);
\item Solution 2: single connection and use of a remote DB (Sec. 8.3.2 of D4.1);
\item Solution 3: double connection and use of a remote DB (Sec. 8.3.3 of D4.1).
\end{itemize}

Sec. 8.3 of D4.1 also describes the different features, in terms of scalability, that the three solutions can provide. For the deployment of CIVIS ICT Platform we adopted Solution 1, because it was suitable for achieving all the fixed goals in the provided project time frame and that is represented in Figure 1.

\subsection{T3.4 Energy Service Context}

This section highlights the lessons learned of business models for emerging social energy initiatives.
In its final deliverable (D6.3) work package 6 concludes with four main types of business models that can be applied to the CIVIS maturity scheme (depicting the development from emerging and promising to established energy initiative). 

\begin{itemize}
\item Efficiency Effects: becoming better at what you're doing - activities that enhance brand awareness and increase customer engagement.
\item Diversification: introducing products and services that are different that the initial domain of the organisation -- for (indirect) social as well as (longer term) financial benefits. 
\item Service Provisioning: facilitating others with available expertise -- providing available services at marginal costs to others.
\item Incubation: enabling innovation outside the organisation -- high(er) risk investment in the hope to achieve a financial or technical return on investment. 
\end{itemize}

Regardless of the specific maturity stage an organisation stage has achieved, of these four business model types, trying to achieve efficiency effects is one that is most applicable to any maturity stage. 

To that effect, the mobile energy app and the community platform that have been developed as part of this work package can be placed in this category. Both of these tools are prime examples of activities that enhance brand awareness and increase customer engagement in order to either make the members of the cooperatives more energy aware (and efficient) or that it convinces more inhabitants to join the cooperative.

As such these developments are (when introduced in the right way) powerful tools that can strengthen a chosen business model enormously. In work package 6 much attention has given to a step-by-step approach for (emerging) energy initiatives and particularly the use of tooling such as the business model canvas (bmc). The developed tools in work package three provides a solution for cooperatives to properly address their (mobile and online) members -- in other words, describing the ``channel'' element of the bmc.

Examples of this have become apparent in surveys with the board members of the testsites in Sweden and Italy. What CIVIS learned is that there is a distinct advantage in having a (ICT) collaboration or (ICT) information solution available to the cooperative members. While it does not come as a surprise that providing members with information on energy generation, consumption and saving does increases awareness -- it is the extent of the impact on awareness that interests us.

While initial uptake of the CIVIS apps has been slow, once people started to use it the awareness of what is happening within the cooperative as a whole is certainly increasing. Though not statistical relevant, several board members have remarked increased feedback from members since making the mobile apps available. For leadership, this kind of feedback and interaction is particularly helpful terms of gauging if certain initiatives have a positive effect within the cooperation. In a sense this contributes more to their understanding than just looking at the amount of website ``hits'' that are available through the IT responsible department or person. It is a combination of context and hard numbers that makes the learning process possible in the first place. 
