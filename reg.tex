\section{User Registration Process}

In general, a user can sign up for a YouPower account at the application's welcome page\footnote{\url{https://app.civisproject.eu/frontend.html}}. In this case, the user can use the Action Suggestion part of the application. For the Housing Cooperative part or the Energy Awareness part of the application, more specific user (and energy) information is needed by the application. There are tailored solutions for the Swedish and the Italian test sites respectively which are presented in the following two subsections. 

\subsection{Swedish Test Site}

For the Swedish test site users, the registration process is as follows. 

\begin{enumerate}
\item A user receives a flyer with a URL that is housing-cooperative-specific, i.e. the housing cooperative which the user belongs to. 

The URL location is managed by KTH (i.e. it is a local Swedish URL) in the format of  \texttt{\small http://civis.proj.kth.se/}\textit{\{unique-link-of-a-housing-coopertive\}}. 

\item Through the URL, a user can reach a YouPower (Swedish user) sign up page that in addition to the inquiry of standard user information also asks for a unique identifier which allows to identify the user's household (and hence the associated energy data) with the user's permission. 

This is similar to but different from the Italian process where Italian test site users get their ApartmentID and HouseholdID beforehand. The Swedish process uses a unique ID of a user's electricity bill to identify different households. (Each user of a household still can register a separate user account at YouPower.) At the Swedish sign up page, there are also descriptions explaining how users can look up their IDs (which might be different for different housing cooperatives). 

\item The Swedish sign up page then asks for the user's household information, e.g. ...??? . 

\item After the registration, the page suggests the user to invite other household members via email.

\item At the end of the process, the user is directed to the housing cooperative view of the YouPower application. 

\end{enumerate}

\subsection{Italian Test Site}
New Trentino users should follow a 2 step process to use the full feature of YouPower.\\
\begin{enumerate}
\item A user will signup from YouPower \href{https://app.civisproject.eu/frontend.html#/welcome/}{home page} by filling up basic information(similar to swedish users). 
Users will then automatically be logged into YouPower after successful registration.
\item In order to access functionalities of the app which are discussed in Section  ~\ref{sec:energydata}, italian users should provide a valid \emph{ContractID} and \emph{Testbed Id}. Contract Id's will then be matched to ApartmentID of the household to access energy data from the backend while the testbed location will be used to fetch community related information like TOU signals in the municipality and  total consumption of the test site. At this step, the \emph{Energy Data} tab of YouPower will be activated for Trentino users.
\end{enumerate}