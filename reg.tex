\section{User Registration Process}

General users

\subsection{Swedish Test Site}

For the Swedish test site users, the registration process is as follows. 

\begin{enumerate}
\item A user receives a flyer with a URL that is housing-cooperative-specific, i.e. the housing cooperative which the user belongs to. 

The URL location is managed by KTH (i.e. it is a local Swedish URL) in the format of  \texttt{\small http://civis.proj.kth.se/}\textit{\{unique-link-of-a-housing-coopertive\}}. 

\item Through the URL, a user can reach a YouPower (Swedish user) sign up page that in addition to the inquiry of standard user information also asks for a unique identifier which allows to identify the user's household (and hence the associated energy data) with the user's permission. 

This is similar to but different from the Italian process where Italian test site users get their ApartmentID and HouseholdID beforehand. The Swedish process uses a unique ID of a user's electricity bill to identify different households. (Each user of a household still can register a separate user account at YouPower.) At the Swedish sign up page, there are also descriptions explaining how users can look up their IDs (which might be different for different housing cooperatives). 

\item The Swedish sign up page then asks for the user's household information, e.g. ...??? . 

\item After the registration, the page suggests the user to invite other household members via email.

\item At the end of the process, the user is directed to the housing cooperative view of the YouPower application. 

\end{enumerate}

\subsection{Italian Test Site}