\section{Conclusion}

The main activities of WP3 in the 3rd year of the CIVIS project were carried out to continue and complete the design and development of the CIVIS platform, called YouPower, to meet the project goal and to suit the context and objective of the test sites' use cases. 
We continued the adaptive agile development process, and expanded upon the results from the 2nd year of the CIVIS project. The design of YouPower is composed of a set of features that can be divided into three interrelated parts: 
\begin{enumerate}
\item Action suggestions, where users are provided with suggestions of  actions that are implementable in everyday household practices. Users are presented with motivators for behavior change (and its process), and means for self-monitoring, self-reflection and social sharing of the pro-environmental behavior change process at individual, household and community levels.

\item Housing cooperatives, where energy management and cooperative energy reduction actions are linked to each cooperative's energy performance (and building level energy information). YouPower allows comparison between housing cooperatives based on the energy performance and actions, and supports knowledge exchange and social sharing and learning among cooperatives in energy reduction actions.

\item Energy awareness, where users are presented with real time and historical energy consumption and production information at individual and comparative (community) level, and are provided with time-of-use signals based on real time and forecasted information about local production from renewable sources and the corresponding energy demand in order to enable load-shifting actions in everyday practice. 

\end{enumerate}

Along the design of those features, we developed ten design guidelines that address a number of design issues emerged from the CIVIS YouPower project. These guidelines are based on literature about pro-environmental intervention and social and environmental psychology combined with our design experience, and can be useful for the design of other energy community and applications for pro-environmental interventions. 

The CIVIS YouPower platform is deployed at the Stockholm (Sweden) and Trento (Italy) test sites:  the first deployment at the Stockholm test site was in Nov 2015 and Trento test site in Mar 2016 correspondingly, and since then the production server had been updated once there were new parts ready for deployment. WP3 uses two ways to monitor YouPower user activities: (1) the API calls to the WP3 CIVIS server are logged into WP3 YouPower database; (2) the important user actions (i.e., user clicks) in the YouPower app are logged  through \textit{MixPanel}\footnote{\url{https://mixpanel.com/}}.
During the 2nd year's CIVIS review meeting in Brussels, the CIVIS project team committed to the YouPower data collection (keep the servers running) for 6 months after the end of the project. This commitment will be carried out by WP3. 

In principle, any interested users (also those outside of the test sites) can sign up the YouPower application. The Action Suggestion part of the application is ready to be used as-is. The Housing Cooperative part is applicable to energy management works of housing cooperatives/associations that exist in a number of EU and non EU countries. The cooperative energy performance feature requires energy consumption data at cooperative level. The Energy Awareness part of the application requires access to users household energy consumption and production (if applicable) data. At the CIVIS test sites, the required energy data at cooperative, household and individual appliance levels are collected by smart meters and smart plugs/sensors. The details about energy data accessibility often differ from case to case, hence requires some adaptation. 

As stated in an earlier deliverable, designing well is not easy \citep{Norman2002}: if a product is truly new, it is unlikely that anyone will know how to design it right at the first time.  
The design and development of YouPower together with user studies at the Swedish and Italian test sites by the CIVIS project formed a good basis for and provided valuable insights into the social dimension of the energy network to achieve sustainable energy goals. This is a first step towards long-term sustainable socio-technical development.  
CIVIS partners are looking for opportunities, i.e. new research projects, to bring the result of this project into further steps. The CIVIS project has its results and the source of the YouPower application open to the public, which may bring future endeavors of interested communities to explore and expand upon the current effort towards promoting sustainable social energy actions. 



