%Survies and statistics show that despite many users are concerned about their energy use, and interested in reducing energy use and saving energy bill, little of them actually know what to do to achieve the goal. 
%Today, consumers still have a vague understanding of the power grid or what they can do with it. This is however very important, because how consumers understand the smart grid will shape how they feel about it, and in turn determines whether they are ready to use it, and how they use it. 
%
%To this end, it is very important how we present energy information (or information or knowledge of the power grid in general) to the users. 
%
%It is clear that we need to provide them with personalized information. And more importantly, we need to give them actionable information. 
%
%If consumers are given useful feedback on how they use energy, and they are given recommendations on how to improve, they will have the chances to make more informed energy choices. 
%
%This may have chances to yield crowdshifting effect, which simply means large-scale, voluntary behavioral change for social good. 
%
%There are many finer details in how to present users with information and how this could affect user behavior. All these are interesting topics to study. 

\subsubsection{YouPower Application Navigation Structure}

The side navigation (nav) of the YouPower front-end application is composed of six items, among which the ``Housing Cooperatives'' (Sec. \ref{sec:brf}) or ``Energy Data'' (Sec. \ref{sec:energydata}) is activated respectively when a user is a member of a housing cooperative (the Swedish case) or after a user authenticated his/her household's account for energy data (production is only for the Italian test site); see for example Figure~\ref{fig:tab}, which depicts an Italian case with an open side navigation drawer. 
Except for the ``Log Out'' item, each nav item is associated with at least one tab item, which in term has different views. Table~\ref{tab:app_nav} presents the navigation structure of the YouPower application. 

\begin{longtable}{ p{3.5cm}  p{3.5cm} p{7.5cm}}
\caption{YouPower app navigation structure}\label{tab:app_nav}\\
\hline
\textbf{Side Nav Items}  &
\textbf{Tab Items}  &
\textbf{Views}  \\ \hline

Actions & Your Actions & Current Actions (and view/add Comments, Like, Share), Completed Actions (and view/add Comments, Like, Share), Suggested Action, Postpone Action, Action Completed (Form), Action Completed (Form) \\ \cline{2-3}
& Household Actions & Invited to Your Household, Pending Invites, Members, Actions (of household members), Invite Household Member (search and invite), Send an Invitation by Email \\ 
%& Community Actions & Community List, Top Action List, Discussions, etc.\\ 
%& Achievements & Achievement List (unlocked/locked), etc.\\  
\hline

Energy Data  & Household Level  & Current Tarif (Trentino only), Current Consumption, Current Production, Historical Production/Consumption Patterns, Forecasted Tarifs (Trentino only), consumption comparison in High and low energy tariff, Consumption comparison with total and average consumption of a test site (Trentino only) \\  \cline{2-3}
& Appliance Level & Consumption Patterns and `last-reading' update (most recent time that consumption data is provided by Energy ICT platform) for each monitored appliances\\  \cline{2-3}
&  Community Level & Total Community Consumption (last month), Total Community Production (last month), Community Energy Balance, Comparison with Benchmark, Total Community consumption in high and low energy Tariff (Trentino only), Daily consumption comparision between Municipalities (Trentino test sites)\\  \hline

Housing Cooperatives & \textit{Name} (of ones own cooperative) & Cooperative Actions, Consumption (Monthly / Yearly, Heating \& Hotwater / Facilities electricity, Compare to neighborhood average / previous year), View/add discussion \\ \cline{2-3}
& Neighboring Cooperatives &  Map/List (of cooperatives and actions) \\  \hline

%Donation (Trentino only) & n/a & Campaign Information, Campaign Status \\ \hline

Settings & Settings & Configurations of actions, Link to Facebook \\  \cline{2-3}
& Personal Profile & Email, Name, Testbed location, Preferred Language, etc.  \\ \cline{2-3}
& Household Profile & Cooperative (if applicable), size, Household Composition, etc.\\  \hline

About & About & Q\&A List, Contact\\ \hline
Log Out &   &  \\ \hline
\end{longtable}


%The ``Action List'' is the index view of ``Your Actions'' tab. It is also the default view after user login. In this case, when the user presses on one of the ``Current Actions'', the app navigates to the ``Action Details'' view. Figure~\ref{fig:action_details} gives an example. 
% 

