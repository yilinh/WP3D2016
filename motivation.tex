\subsection{Motivation: Bridging the Attitude-Behavior Gap}

Environmentally significant behaviours should be understood as relatively inconspicuous actions performed in the context of everyday life \citep{Burgess2008}. People consume energy through many daily practices and routines in households \citep{Burgess2008,Hargreaves2010, Fehrenbacher2011,Burchell2014}. How and to what extent these daily actions affect domestic energy use and in turn the environment are not always readily apparent to average consumers  \citep{Burgess2008,Delmas2013}. This knowledge deficit poses significant constraints for consumers to perform and engage in energy conservation (and energy efficiency) behaviors \citep{Schultz2002,Burchell2014}. Acquiring such information is often costly  \citep{Delmas2013}. 

A rich body of empirical research suggests that relevant information tends to result in higher knowledge levels but not necessarily in behavior changes or energy savings  \citep{Abrahamse2005,Delmas2013,Burchell2014,Asensio2015}. Despite growing environmental awareness and articulated preference for ``green'' lifestyles, people's environmental values and attitudes often fail to materialize in actual actions and behavior changes, from energy conservation, to recycling,  to the purchase of green products  \citep{Schultz2002,Abrahamse2005,Claudy2013}. This imparity is commonly referred to as the attitude-behavior gap or the value-action gap  \citep{Blake1999,Kollmuss2002,Claudy2013}. 

Although there is no single framework or theory that provides definitive explanations for the attitude-behavior gap  \citep{Kollmuss2002,Schultz2014}, literature provides suggestions that shed some light on this issue. People perform (or do not perform) certain pro-environmental actions for many reasons  \citep{Schultz2002}. The reasons for acting are often referred to as motives or motivation  \citep{Parfit1997,Moisander2007}. A distinction can be made between \textit{primary motives} and \textit{selective motives}  \citep{Kollmuss2002,Moisander2007}. Primary motives influence decisions to engage (or not to engage) in a whole class of actions or behaviors. For example, \textit{Do I want to bike to work (in general)?} They can be understood as general attitudes towards certain actions. Selective motives influence decisions on specific actions. For example, \textit{(It is cold and raining.) Do I want to bike to work (now)?} They have direct positive or negative impact on the actions. In this sense, primary motives, such as altruistic and social values which build up attitudes, have no direct influence on specific actions. They are often covered up by more immediate selective motives, which evolve around personal and everyday needs and context such as comfort, practicality and complexities in everyday life  \citep{Kollmuss2002,Berthou2013,Selvefors2015}. 

The countervailing influences of context-specific reasons for or against specific actions (that is, the selective motives aforementioned) are strong antecedents of one's decisions on the actions  \citep{Claudy2013}. In particular, a decision is often more strongly influenced by reasons against the action  \citep{Claudy2013,Berthou2013}. This means, one can decide not to or fail to perform pro-environmental actions (because of context-specific reasons against the actions) even if one holds pro-environmental values and attitudes. For example, load-shifting of electricity use by doing laundry at night is not an option for shared laundry facilities that are only open during daytime  \citep{Entwistle2015}; a tenant may depend on the landlord for certain energy reduction actions (that involve investments) to be taken  \citep{Dillahunt2010}. In general, one's abilities and willingness to take energy conservation actions are constrained by the context-specific reasons in everyday life. 
In many cases, people act habitually or routinely rather than making reasoned choices  \citep{Steg2009,Berthou2013}. Habits are learned sequences of actions that have become automatic responses to specific cues, and are functional in obtaining certain goals or end-states  \citep{Verplanken1999}. Those who have tried to change a habit, even in a minor way, would discover how difficult it is even if the new behavior has distinct advantages over the old one  \citep{Kollmuss2002}. When an individual wants to establish a new behavior, the person has to practice it  \citep{Kollmuss2002}. One might be perfectly willing to change certain behavior but still not do so because the person does not persist enough in practicing the new behavior until it has become a habit  \citep{Kollmuss2002}. A sustained behavior change requires learning a new habit  \citep{Dillahunt:2009:GEU:1620545.1620583}. 

To bridge the attitude-behavior gap, household energy conservation (and load-shifting) behavior interventions can be geared towards the facilitation of the behavior change process in everyday life. The goal of this process is to motivate consumers to learn and practice new energy consumption behaviors until those behaviors become new habits that are embedded in the specific context of their everyday life. In particular, this means that (1) consumers need to be provided with accurate information about actionable suggestions on how to achieve potential energy conservation (and load-shifting) , and (2) the intervention design shall also have means to motivate consumers to voluntarily practice and repeat the energy conservation (and load-shifting) actions in the specific context of their everyday life.  
